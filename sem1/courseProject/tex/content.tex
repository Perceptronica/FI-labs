\pagestyle{main}
\begin{large}
	\textbf{\DocName \ \Number \ по курсу \CourseName}
\end{large}	
\\
Студент группы: \Group, \ \Name, № по списку: \StudentNumber, контакты: \Contacts
\begin{flushright}
	Работа выполнена: \CompletionDate \\
	Преподаватель: \Lecturer \\
	Входной контроль знаний с оценкой: \\
	Отчет сдан \ReportDate, итоговая оценка \Mark \\
	Подпись преподавателя: \Signature
\end{flushright}
\begin{enumerate}
	\item \textbf{Тема:} \ \Theme
	\item \textbf{Цель работы: } \Target
	\item \textbf{Задание} \textit{(вариант № \StudentNumber)}: \Task
	\item \textbf{Оборудование:} \\ \Equipment
	\item \textbf{Программное обеспечение (лабораторное):} \\ \Software
	\item \textbf{Идея, метод, алгоритм} 
	\begin{footnotesize}
		решения задачи (в формах: словесной, псевдокода, графической [блок-схема, диаграмма, рисунок, таблица] или формальные спецификации с пред- и постусловиями)
	\end{footnotesize} 
	\\ \Idea
	\item \textbf{Сценарий выполнения работы}
	\begin{footnotesize}
		[план работы, первоначальный текст программы в черновике (можно на отдельном листе) и тесты либо соображения по тестированию]
	\end{footnotesize}
	\Plan
	\\
	\textit{Пункты 1-7 отчета составляются строго до начала лабораторной работы.}
	\begin{flushright}
		\textit{Допущен к выполнению работы.} \textbf{Подпись преподавателя:} 
	\end{flushright}
	\item \textbf{Распечатка протокола}
	\begin{footnotesize}
		(подклеить листинг окончательного варианта программы с тестовыми примерами, подписанный
		преподавателем):
	\end{footnotesize}
	\Protocol
	\item \textbf{Дневник отладки} 
	\begin{footnotesize}
		должен содержать дату и время сеансов отладки и основные события (ошибки в сценарии и программе,
		нестандартные ситуации) и краткие комментарии к ним. В дневнике отладки приводятся сведения об использовании других ЭВМ,
		существенном участии преподавателя и других лиц в написании и отладке программы:
	\end{footnotesize}
	\\
	\Debugging
	\item \textbf{Замечания автора} по существу работы: 
	\item \textbf{Выводы:} \Conclusions
	\\
	Недочёты при выполнении задания могут быть устранены следующим образом: \Defects
	\begin{flushright}
		Подпись студента:
	\end{flushright}
\end{enumerate}