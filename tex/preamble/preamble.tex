\documentclass[a4paper, 12pt]{report}

% Поддержка русского языка
\usepackage[english, russian]{babel}
\usepackage[T2A]{fontenc}
\usepackage[utf8]{inputenc}
\usepackage{indentfirst}
\usepackage{hyperref}

\usepackage{amsmath, amsfonts, amssymb, amsthm, mathtools}

%% Оформление страницы
\usepackage{extsizes}     % Возможность сделать 14-й шрифт
\usepackage{geometry}     % Простой способ задавать поля
\usepackage{setspace}     % Интерлиньяж
\usepackage{enumitem}     % Настройка окружений itemize и enumerate
\setlist{leftmargin=25pt} % Отступы в itemize и enumerate

\geometry{top=25mm}    % Поля сверху страницы
\geometry{bottom=30mm} % Поля снизу страницы
\geometry{left=20mm}   % Поля слева страницы
\geometry{right=20mm}  % Поля справа страницы

\setlength\parindent{15pt}        % Устанавливает длину красной строки 15pt
\linespread{1.3}                  % Коэффициент межстрочного интервала
%\setlength{\parskip}{0.5em}      % Вертикальный интервал между абзацами
%\setcounter{secnumdepth}{0}      % Отключение нумерации разделов
%\setcounter{section}{-1}         % Нумерация секций с нуля
\usepackage{multicol}			  % Для текста в нескольких колонках
\usepackage{soulutf8}             % Модификаторы начертания

\newcommand{\nc}{\newcommand}

\nc{\Group}{\textbf{М8О-101Б-22}}
\nc{\Name}{\textbf{Кабанов Антон Алексеевич}}
\nc{\StudentNumber}{\textbf{7}}

\nc{\email}{\href{mailto:anton1258kab@gmail.com}{\textbf{anton1258kab@gmail.com}}}
\nc{\Contacts}{\email}

\nc{\Lecturer}{каф. 806 Крылов Сергей Сергеевич}

\nc{\Processor}{\textbf{AMD Ryzen 5500U (6-ядерный, @2.1 ГГц})}
\nc{\RAM}{\textbf{15345 Мб}}
\nc{\ROM}{\textbf{479.9 Гб}}
\nc{\Screen}{\textbf{встроенный, IPS, 2160x1440, @60 Гц}}

\nc{\OSFamily}{\textbf{GNU/Linux}}
\nc{\OSName}{\textbf{Manjaro Linux}}
\nc{\OSVersion}{\textbf{5.15.76-1-MANJARO}}
\nc{\Shell}{\textbf{bash}}
\nc{\ShellVersion}{\textbf{5.1.16}}

\nc{\ProgrammingLanguage}{\textbf{C}}
\nc{\TextEditor}{\textbf{emacs, vim (neovim)}}
\nc{\OSUtilities}{\textbf{pwd, who, ls, cd, mv, cp, rm, rmdir, mkdir, cat, whoami, man}}
\nc{\AppliedSystems}{\textbf{touch, echo, pacman, chmod, date, lsblk, gnuplot, emacs, nvim}}
\nc{\FileLocation}{\textbf{/home/void/Документы/FI-labs}}

\nc{\Equipment}{
	\textit{Оборудование ПЭВМ студента, если использовалось:} \\
	Процессор \Processor \ с ОП \RAM, ТТН \ROM. Монитор \Screen.
}

\nc{\Software}{
	\textit{Программное обеспечение ЭВМ студента, если использовалось:} \\
	Операционная система семейства \OSFamily, наименование \OSName \ версия \OSVersion, интерпретатор
	команд \Shell \ версия \ShellVersion. \\
	Система программирования: \ProgrammingLanguage \\
	Редактор текстов: \TextEditor \\
	Утилиты операционной системы: \OSUtilities \\
	Прикладные системы и программы: \AppliedSystems \\
	Местонахождение и имена файлов программ и данных на домашнем компьютере: \FileLocation
}

\usepackage{titleps}
\newpagestyle{main}{
	\setheadrule{0.1pt}
	\sethead{\DocName \ \Number}{}{}
	\setfootrule{0.1pt}
	\setfoot{}{}{\thepage}
}

\usepackage{xcolor}
\usepackage{listings}
\lstset{basicstyle=\ttfamily,
	showstringspaces=false,
	commentstyle=\color{red},
	keywordstyle=\color{blue}
}